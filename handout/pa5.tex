% CSCI3753 - Operating Systems
% Spring 2012
% Programming Assignment 5
% By Andy Sayler (4/18/12) <www.andysayler.com>

\documentclass[12pt]{article}

\usepackage[text={6.5in, 9in}, centering]{geometry}
\usepackage{graphicx}
\usepackage{url}
\usepackage{listings}
\usepackage{hyperref}

\lstset{
  language={c},
  basicstyle=\footnotesize,
  numbers=left,
  numberstyle=\tiny,
  stepnumber=1,
  numbersep=5pt,
  showspaces=false,
  showstringspaces=false,
  showtabs=false,
  tabsize=4,
  captionpos=b,
  breaklines=true,
  breakatwhitespace=false,
  frame=single,
  frameround=tttt
}

\hypersetup{
    colorlinks,
    citecolor=black,
    filecolor=black,
    linkcolor=black,
    urlcolor=black
}

\newenvironment{packed_enum}{
\begin{enumerate}
  \setlength{\itemsep}{1pt}
  \setlength{\parskip}{0pt}
  \setlength{\parsep}{0pt}
}{\end{enumerate}}

\newenvironment{packed_item}{
\begin{itemize}
  \setlength{\itemsep}{1pt}
  \setlength{\parskip}{0pt}
  \setlength{\parsep}{0pt}
}{\end{itemize}}

\newenvironment{packed_desc}{
\begin{description}
  \setlength{\itemsep}{1pt}
  \setlength{\parskip}{0pt}
  \setlength{\parsep}{0pt}
}{\end{description}}

\title{Programming Assignment 5:\\An Encrypted Filesystem}
\author{
  CSCI 3753 - Operating Systems\\
  University of Colorado at Boulder\\
  Spring 2012\\
  By Andy Sayler and Junho Ahn and Richard Han\\
  Adopted from Assignment by Chris Wailes
}
\date{\emph{Due Date: Friday, April 27th, 2012 11:55pm}}

\begin{document}

\maketitle

\section{Assignment Introduction}


\section{Your Task}

\subsection{Dependencies}
This assignment has severl dependines that must be isnatlled in order
for teh rpeovided code to build correctly. On Debian or Ubuntu, start
by running \texttt{sudo apt-get update} to update your package
list. Then run \texttt{sudo apt-get install <package(s)>} to install the
following packages:

\begin{packed_item}
\item \texttt{fuse-utils}
\item \texttt{libfuse-dev}
\item \texttt{openssl}
\item \texttt{libssl-dev}
\item \texttt{libssl-doc} (optional)
\item \texttt{libssl1.0.0} or \texttt{libssl0.9.8}
\item \texttt{attr}
\item \texttt{attr-dev}
\end{packed_item}

You will need a working Internet connection in order to insure these
packages install correctly. Note that you can also specify multiple
packages in a single \texttt{sudo apt-get install <package(s)>}
call. Some packages may have their own dependencies, but
\texttt{apt-get} will automatically take care of installing these for you.

\section{What's Included}

We provide some code and examples to help get you started.
Feel free to use it as a jumping off point (appropriately cited).

\begin{packed_item}
\item {\bf Makefile} A GNU Make makefile to build all the code listed
  here.
\item {\bf README} As the title so eloquently instructs: read
  it. Provides usage instructions and examples for files listed here.
\item {\bf fusehello.c} A basic "Hello World" FUSE example. See
  \texttt{README} for usage instructions.
\item {\bf fusexmp.c} A basic FUSE mirrored filesystem example that
  mirrors the root directory (/) and supports most standard
  operations. See \texttt{README} for usage instructions.
\item {\bf xattr-util.c} A basic extended attribute manipulation
  program. Provides an example of proper Linux xattr use.
  See \texttt{README} for usage instructions.
\item {\bf aes-crypt-util.c} A basic AES encryption program using the
  local aes-crypt library (see \texttt{aes-crypt.h}) and the OpenSSL EVP
  API\cite{openssl-evp}. See \texttt{README} for usage instructions.
\item {\bf aes-crypt.h} A basic AES file-pointer centric encryption
  library interface. Implemented in \texttt{aes-crypt.c}.
\item {\bf aes-crypt.c} A basic AES file-pointer centric encryption
  library implementation. Uses the OpenSSL EVP API \cite{openssl-evp}.
\end{packed_item}

\section{What You Must Provide}

When you submit your assignment, you must provide the following as a
single archive file:
\begin{packed_enum}
\item A copy of your Encrypted Filesystem FUSE code
\item A copy of any supporting code used by your filesystem
\item A makefile that builds any necessary code
\item A README explaining how to build and run your code
\end{packed_enum}

\section{Grading}

40\% of you grade will be based on implementing a filesystem that
meets the following criteria. You will be expected to provide
functional proof of the following criteria during your grading
session. The rubric below shows grading criteria:

\begin{packed_item}
\item {\bf +10 points:} Filesystem properly mirrors target directory
  specified at mount time.
\item {\bf +10 points:} Filesystem uses extended attributes to differentiate
  between encrypted and unencrypted files.
\item {\bf +10 points:} Filesystem can transparently read and write
  securely encrypted files 
\item {\bf +10 points:} Filesystem can transparently read and update unencrypted
  files
\end{packed_item}

If your code does not build or run without errors, you will not receive
any credit on the objective portion (40\%) of your assignment.

If your code generates warnings when building under gcc on the VM
using \texttt{-Wall} and \texttt{-Wextra} you will be penalized 1
point per warning. In addition, to receive full credit your submission must:
\begin{packed_item}
\item Meet all requirements elicited in this document
\item Code must adhere to good coding practices.
\item Code must be submitted to Moodle prior to due date.
\end{packed_item}

The other 60\% of your grade will be determined via your grading
interview where you will be expected to explain your work and answer
questions regarding it and any concepts related to this assignment.

\section{Obtaining Code}
The starting code for this assignment is available on the Moodle and
on github. If you would like practice using a version control system,
consider forking the code from github. Using the github code is not
a requirement, but it will help to insure that you stay up to date
with any updates or changes to the supplied codebase. It is also
good practice for the kind of development one might expect to do in
a professional environment. And since your github code can be easily
shared, it can be a good way to show off your coding skills to
potential employers and other programmers.

Github code may be forked from the project page here:\\
\url{https://github.com/asayler/CU-CS3753-2012-PA5}.

\section{Resources}
Refer to your textbook and class notes on the Moodle for an overview
of filesystems.

If you require a good C language reference, consult K\&R\cite{K+R}. If
you need an updated C99 reference see Harbison \& Steele\cite{H+S}.

The Internet\cite{tubes} is also a good resource for finding
information related to solving this assignment.

You may wish to consult the man pages for the following items, as they
will be useful and/or required to complete this assignment. Note that
the first argument to the ``man'' command is the chapter, insuring
that you access the appropriate version of each man page. See
\texttt{man man} for more information. Not all of these man pages are
installed be default. Install the previously discussed dependencies or
consult an online man page repository if you can not locate a specific
man page on your system.

\begin{packed_item}
\item \texttt{man 1 make}
\item \texttt{man 1 fusermount}
\item \texttt{man 5 attr}
\item \texttt{man 2 setxattr}
\item \texttt{man 2 getxattr}
\item \texttt{man 2 listxattr}
\item \texttt{man 2 removexattr}
\item \texttt{man 3 EVP}
\item \texttt{man 3 EVP\_CipherUpdate}
\item \texttt{man 3 crypto}
\item Many of the system calls used in the FUSE examples also have man pages
\end{packed_item}

In addition, you may find a number of the references in the bibliography helpful.

\begin{thebibliography}{3}

\bibitem{freedesktop-xattr} freedesktop.org.
  \newblock \emph{Guidelines for extended attributes}.
  \newblock \url{http://www.freedesktop.org/wiki/CommonExtendedAttributes}.

\bibitem{fuse-website} FUSE.
  \newblock \emph{Filesystems in Userspace}.
  \newblock \url{http://fuse.sourceforge.net/}.

\bibitem{fuse-refs} FUSE.
  \newblock \emph{Fuse Doxygen API Reference}.
  \newblock \url{http://fuse.sourceforge.net/doxygen/index.html}.

\bibitem{fuse-wiki} FUSE.
  \newblock \emph{Fuse Wiki}.
  \newblock \url{http://sourceforge.net/apps/mediawiki/fuse/index.php?title=Main_Page}.

\bibitem{H+S} Harbison, Samuel and Steele, Guy.
  \newblock \emph{C: A Reference Manual}.
  \newblock Fifth Edition: 2002.
  \newblock Prentice Hall: New Jersey.

\bibitem{K+R} Kernighan, Brian and Dennis, Ritchie.
  \newblock \emph{The C Programming Language}.
  \newblock Second Edition: 1988.
  \newblock Prentice Hall: New Jersey.

\bibitem{openssl-website} OpenSSL.
  \newblock \emph{Cryptography and SSL/TLS Toolkit}.
  \newblock \url{http://www.openssl.org/}.

\bibitem{openssl-docs} OpenSSL.
  \newblock \emph{OpenSSL Documents}.
  \newblock \url{http://www.openssl.org/docs/}.

\bibitem{openssl-evp} OpenSSL.
  \newblock \emph{OpenSSL EVP Documentation}.
  \newblock \url{http://www.openssl.org/docs/crypto/EVP_EncryptInit.html}.

\bibitem{pillai-aes} Pillai, Saju.
  \newblock \emph{Openssl AES encryption example}.
  \newblock Decemper 9th, 2008.
  \newblock \url{http://saju.net.in/blog/?p=36}.

\bibitem{pfeiffer-fuse} Pfeiffer, Joseph.
  \newblock \emph{Writing a FUSE Filesystem: a Tutorial}.
  \newblock January 10th, 2011.
  \newblock \url{http://www.cs.nmsu.edu/~pfeiffer/fuse-tutorial/}.

\bibitem{tubes} Stevens, Ted.
  \newblock \emph{Speech on Net Neutrality Bill}.
  \newblock 2006.
  \newblock \url{http://youtu.be/f99PcP0aFNE}.

\end{thebibliography}

\end{document}  
